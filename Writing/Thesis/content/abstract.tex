% !TEX root = ../my-thesis.tex
%
\pdfbookmark[0]{Abstract}{Abstract}
\chapter*{Abstract}
\label{sec:abstract}
\vspace*{-10mm}

Although the amount of data available worldwide is increasing, the portion of analyzed data remains low. This problem is likely caused by a lack of experts in the field that are able to carry out an analysis effectively, as it is a difficult task for a layperson. Especially regarding the challenge of classification there exist many algorithms with varying performances from data set to data set - there is no one best algorithm that can always be recommended. One response to this problem is automated machine learning (AutoML): tools that recommend a learning algorithm for a new data set that they estimate to perform best on this particular data set. Current AutoML approaches, however, often search for a fitting algorithm without explicitly taking into account how properties of the data set in question might influence the performances of classifiers. For the most part, they also suggest a single algorithm only. In comparison to the recommendation of a ranking of classifiers this has the disadvantage that no alternatives to the suggested algorithm are presented, which, due to a certain inaccuracy of the prediction, may have an equal or even better performance than the first suggestion regarding the examined data set. The problem considered in this thesis is thus the construction of a ranking of classification algorithms in relation to properties of a data set. Two different approaches that aim to solve this problem are presented and compared. The first alternative works on the basis of regression models: one regression model for each classifier predicts a performance value for that classifier. On the basis of the predicted performance values, a ranking then is constructed. In contrast, for the second approach, a preference-based model directly learns and returns rankings. In the evaluation of the two alternatives, it became apparent that the regression-based alternative is superior to the preference-based approach. However, the tested preference models also achieved a significant advantage compared to the static baseline used in the evaluation, at least in certain scenarios. 

\vspace*{20mm}
\newpage

{\usekomafont{chapter}Abstract (German)}\label{sec:abstract-german} \\

% Motivation
Die weltweit verfügbare Menge an Daten steigt immer weiter an, und doch bleibt der Anteil der tatsächlich analysierten Daten gering. Dieses Problem entsteht, da nicht genug Experten verfügbar sind, um diese Analyse auszuführen, und es für Laien schwierig ist, das richtige Verfahren auszuwählen. Besonders im Bereich der Klassifikation gibt es viele Algorithmen, deren Performanz jedoch von Datensatz zu Datensatz variieren kann - es gibt also keinen einen besten Algorithmus, der immer eine gute Wahl darstellt. Eine Antwort auf diese Problem ist automatisiertes Machine Learning. Daher werden in dieser Arbeit zwei verschiedene Ansätze, welche für einen neuen Datensatz ein Ranking von Klassifizierungsalgorithmen aufgrund von Eigenschaften des Datensatzes vorhersagt, vorgestellt und miteinander verglichen. Der erste der beiden Ansätze arbeitet auf Grundlage von Regressionsmodellen; ein Ranking wird hier konstruiert indem ein einzelnes Regressionsmodell für jeden Klassifizierungsalgorithmus einen Performanzwert vorhersagt, auf deren basierend dann das Ranking erstellt wird. Der zweite Ansatz beschäftigt sich mit Präferenzmodellen, welche direkt Rankings lernen und zurückgeben. In der Evaluation stellte sich heraus, dass der Regressionsansatz dem Präferenzmodell überlegen ist; das beste der getesteten Regressionsmodelle, basierend auf dem Random Forest Algorithmus, erreicht einen Wert von Kendalls Tau von bis zu 0,495 und Performanzverlustwerte von durchschnittlich nur 1,308 Prozentpunkten. Jedoch sind auch manche der getesten Präferenzmodelle in einigen Punkten wenigstens der zum Vergleich benutzten Baseline überlegen. Daher kann die Schlussfolgerung gezogen werden, dass wenigstens einige der benutzen Eigenschaften der Datensätze mit den Performanzwerten der Klassifizierungsalgorithmen zusammenhängen, und dieser Zusammenhang durch Regressionsmodelle sowohl als auch Präferenzmodelle ausgenutzt werden kann, wobei jedoch der Regressionsansatz letzterem im Rahmen der durchgeführten Evaluation überlegen ist.