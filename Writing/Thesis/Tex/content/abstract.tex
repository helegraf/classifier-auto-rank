\pdfbookmark[0]{Abstract}{Abstract}
\chapter*{Abstract}
\label{sec:abstract}
\vspace*{-10mm}

As the demand for machine learning functionality increases dramatically, automated machine learning (AutoML) aims to (partially) automate the creation of machine learning applications. Current AutoML approaches, however, often search for a single fitting learning algorithm for a new data set without explicitly considering how properties of the data set influence algorithm performances. Furthermore, in comparison to recommending a ranking of learning algorithms, this has the disadvantage that no alternatives are presented. Due to a certain inaccuracy of the prediction, these may have an equal or even better performance than the first suggestion regarding the examined data set.

In this thesis, the problem of predicting a ranking of classification algorithms based on properties of a data set is considered. To this end, two approaches are presented and compared. The first uses regression models to predict the accuracy of each classifier on a new data set, and then constructs a ranking from these predictions. Alternatively, a preference-based model directly learns and returns rankings. In an experimental study, it is shown that the regression-based approach performs superiorly to the preference-based alternative. However, the latter still outperforms a static ranking based on how often a classifier has performed best on previously considered data sets.

\vspace*{20mm}
\newpage

{\usekomafont{chapter}Abstract (German)}\label{sec:abstract-german} \\

Da die Nachfrage nach Funktionalität im Bereich des maschinellen Lernens dramatisch ansteigt, hat automatisiertes maschinelles Lernen (AutoML) das Ziel, die Erstellung von Anwendungen im Kontext des maschinellen Lernens (teilweise) zu automatisieren. Aktuelle Ansätze dieser Art suchen jedoch oft nach einem Lern-algorithmus für einen neuen Datensatz, ohne explizit zu berücksichtigen, wie Eigenschaften des Datensatzes die Performanz der Algorithmen beeinflussen können. Außerdem hat dies im Vergleich zu dem Vorschlag eines Rankings von Lernalgorithmen den Nachteil, dass keine Alternativen präsentiert werden. Diese können allerdings, aufgrund der Ungenauigkeit der Vorhersagen, gleichwertig oder sogar bessere Genauigkeiten auf diesem Datensatz haben.

In dieser Arbeit wird das Problem betrachtet, ein Ranking von Klassifizierungsalgorithmen basierend auf den Eigenschaften eines Datensatzes zu erstellen. Dafür werden zwei Ansätze vorgestellt und verglichen. Der erste benutzt Regressionsmodelle,  um die Genauigkeit jedes Klassifizierers für den neuen Datensatz vorherzusagen, und erzeugt dann basierend auf den Vorhersagen ein Ranking. Alternativ lernt ein Präferenzmodell direkt Rankings und gibt sie zurück. In einer experimentellen Studie wird gezeigt, dass der regressionsbasierte Ansatz eine bessere Performanz aufweist als die präferenzbasierte Alternative. Allerdings ist letztere noch einem statischen Ranking überlegen, welches darauf basiert, wie oft ein Klassifizierer die beste Performanz auf zuvor betrachteten Datensätzen hatte. 
